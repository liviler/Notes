%% This template can be used to write a paper for
%% Computer Physics Communications using LaTeX.
%% For authors who want to write a computer program description,
%% an example Program Summary is included that only has to be
%% completed and which will give the correct layout in the
%% preprint and the journal.
%% The `elsarticle' style is used and more information on this style
%% can be found at 
%% http://www.elsevier.com/wps/find/authorsview.authors/elsarticle.
%%
%%
% \documentclass[preprint,12pt]{elsarticle}

%% Use the option review to obtain double line spacing
%%\documentclass[preprint,review,12pt]{elsarticle}

%% Use the options 1p,twocolumn; 3p; 3p,twocolumn; 5p; or 5p,twocolumn
%% for a journal layout:
%% \documentclass[final,1p,times]{elsarticle}
%% \documentclass[final,1p,times,twocolumn]{elsarticle}
\documentclass[final,3p,times]{elsarticle}
%% \documentclass[final,3p,times,twocolumn]{elsarticle}
%% \documentclass[final,5p,times]{elsarticle}
%% \documentclass[final,5p,times,twocolumn]{elsarticle}

%% if you use PostScript figures in your article
%% use the graphics package for simple commands
%% \usepackage{graphics}
%% or use the graphicx package for more complicated commands
%% \usepackage{graphicx}
%% or use the epsfig package if you prefer to use the old commands
%% \usepackage{epsfig}

%% The amssymb package provides various useful mathematical symbols
\usepackage{amssymb}
%% The amsthm package provides extended theorem environments
%% \usepackage{amsthm}

%% The lineno packages adds line numbers. Start line numbering with
%% \begin{linenumbers}, end it with \end{linenumbers}. Or switch it on
%% for the whole article with \linenumbers after \end{frontmatter}.
%% \usepackage{lineno}

%% natbib.sty is loaded by default. However, natbib options can be
%% provided with \biboptions{...} command. Following options are
%% valid:

%%   round  -  round parentheses are used (default)
%%   square -  square brackets are used   [option]
%%   curly  -  curly braces are used      {option}
%%   angle  -  angle brackets are used    <option>
%%   semicolon  -  multiple citations separated by semi-colon
%%   colon  - same as semicolon, an earlier confusion
%%   comma  -  separated by comma
%%   numbers-  selects numerical citations
%%   super  -  numerical citations as superscripts
%%   sort   -  sorts multiple citations according to order in ref. list
%%   sort&compress   -  like sort, but also compresses numerical citations
%%   compress - compresses without sorting
%%
%% \biboptions{comma,round}

% \biboptions{}

%% This list environment is used for the references in the
%% Program Summary
%%
\newcounter{bla}
\newenvironment{refnummer}{%
\list{[\arabic{bla}]}%
{\usecounter{bla}%
 \setlength{\itemindent}{0pt}%
 \setlength{\topsep}{0pt}%
 \setlength{\itemsep}{0pt}%
 \setlength{\labelsep}{2pt}%
 \setlength{\listparindent}{0pt}%
 \settowidth{\labelwidth}{[9]}%
 \setlength{\leftmargin}{\labelwidth}%
 \addtolength{\leftmargin}{\labelsep}%
 \setlength{\rightmargin}{0pt}}}
 {\endlist}

\journal{Computer Physics Communications}

\begin{document}

\begin{frontmatter}

%% Title, authors and addresses

%% use the tnoteref command within \title for footnotes;
%% use the tnotetext command for the associated footnote;
%% use the fnref command within \author or \address for footnotes;
%% use the fntext command for the associated footnote;
%% use the corref command within \author for corresponding author footnotes;
%% use the cortext command for the associated footnote;
%% use the ead command for the email address,
%% and the form \ead[url] for the home page:
%%
%% \title{Title\tnoteref{label1}}
%% \tnotetext[label1]{}
%% \author{Name\corref{cor1}\fnref{label2}}
%% \ead{email address}
%% \ead[url]{home page}
%% \fntext[label2]{}
%% \cortext[cor1]{}
%% \address{Address\fnref{label3}}
%% \fntext[label3]{}


\title{A Python package for the commutator of many-body operators}


%% use optional labels to link authors explicitly to addresses:
%% \author[label1,label2]{<author name>}
%% \address[label1]{<address>}
%% \address[label2]{<address>}

\author[a]{Y.Li\corref{author}}
% \author[a,b]{Second Author}
% \author[b]{Third Author}

\cortext[author] {Corresponding author.\\\textit{E-mail address:} liyi226@mail2.sysu.edu.cn}
\address[a]{Address XXXX}
% \address[b]{Second Address}

\begin{abstract}
%% Text of abstract

In recent years, the In-Medium Similarity Renormalization Group (IMSRG) has been
making great progress in ab initio nuclear many-body theory. At present, the many-body
operator is generally truncated to two-body in the IMSRG. In order to extend the many-
body truncation to the normal-ordered three-body level, it’s necessary to develop the IM-
SRG(3), which involves a great number of complex calculation processes of many-body
operators. Therefore, writing a computer program to achieve auto-calculation would help
researchers reduce the cost of time and energy on such processes.

To achieve the program, it is necessary to translate the theoretical calculation process
into a series of algorithms and apply these algorithms to the calculation of symbols.Under
the Python platform, based on the computer algebra system provided by SymPy, this
Python package, called combo, can create objects of related classes to indicate the indi-
cators and operators. By applying these objects to a series of algorithms, the combo can
calculate the commutator of two arbitrary many-body operators. Moreover, after solv-
ing the problem above, an interface function matching the Angular Momentum Coupling
Program (AMC) is written and can give the result that coupled the angular momentum.

Specifically, four modules in this package can: a) calculate the product and com-
mutator of two arbitrary many-body operators, b) apply symbols’ rule c) regulate the
expressions and provide some auxiliary functions. d) process the expressions for output.

Based on the functions above, the combo package not only can reproduce the rele-
vant calculational process in the IMSRG but also lays the foundation for the development
of IMSRG(3).This paper introduces the function of each module of this package and il-
lustrates the algorithm’s workflow in some logic diagrams.
\end{abstract}

\begin{keyword}
%% keywords here, in the form: keyword \sep keyword
Python \sep IMSRG\sep commutator of normal-ordered operators\sep combo

\end{keyword}

\end{frontmatter}

%%
%% Start line numbering here if you want
%%
% \linenumbers

% All CPiP articles must contain the following
% PROGRAM SUMMARY.

{\bf \noindent Program summary}
  %Delete as appropriate.

\begin{small}
\noindent
{\em Program Title:} combo                                      \\
{\em CPC Library link to program files:}  \\
{\em Developer's repository link:} https://github.com/liviler/combo \\
{\em Code Ocean capsule:}\\
{\em Licensing provisions:} GPLv3\\
{\em Programming language:} Python 3.10                                 \\
{\em External routines:} sympy  \\
% {\em Supplementary material:}                                  \\
  % Fill in if necessary, otherwise leave out.
% {\em Journal reference of previous version:}*                  \\
  %Only required for a New Version summary, otherwise leave out.
% {\em Does the new version supersede the previous version?:}*   \\
  %Only required for a New Version summary, otherwise leave out.
% {\em Reasons for the new version:*}\\
  %Only required for a New Version summary, otherwise leave out.
% {\em Summary of revisions:}*\\
  %Only required for a New Version summary, otherwise leave out.
{\em Nature of problem:}
The IMSRG achieves success in nuclear physics research.
However, the works to calculate the commutator of normal-ordered operators is  error-prone and would cost a lot of time. And the length of result will grows rapidly as a factorial function.\\
{\em Solution method:}
To calculate the product and commutator of normal-ordered operators, combo uses Wick's theorem to match all possible indicators combinations with the operators and ignore the inappropriate combinations.
Afterwards, combo can splite the previous expression and regulate it. And you can display all attributes that store the result of your behaviour.
Moreover, an interface function is given to match the Angular Momentum Coupling
Program (AMC).
% {\em Additional comments including restrictions and unusual features:}\\
  %Provide any additional comments here.
  %  \\


% \begin{thebibliography}{0}
% \bibitem{1}Reference 1         % This list should only contain those items referenced in the                 
% \bibitem{2}Reference 2         % Program Summary section.   
% \bibitem{3}Reference 3         % Type references in text as [1], [2], etc.
%                                % This list is different from the bibliography at the end of 
%                                % the Long Write-Up.
% \end{thebibliography}
% * Items marked with an asterisk are only required for new versions
% of programs previously published in the CPC Program Library.\\
\end{small}


%% main text
\newpage
\input{section/Introduction.tex}

\input{section/Background-theory.tex}

\section{Implementation}
\subsection{Algorithm}

\subsection{Code structure}

\subsection{Usage}

\input{section/Conclusion.tex}


%% The Appendices part is started with the command \appendix;
%% appendix sections are then done as normal sections
%% \appendix

%% \section{}
%% \label{}

%% References
%%
%% Following citation commands can be used in the body text:
%% Usage of \cite is as follows:
%%   \cite{key}         ==>>  [#]
%%   \cite[chap. 2]{key} ==>> [#, chap. 2]
%%

%% References with bibTeX database:
\cite{}
\bibliographystyle{elsarticle-num}
\bibliography{main.bib}

%% Authors are advised to submit their bibtex database files. They are
%% requested to list a bibtex style file in the manuscript if they do
%% not want to use elsarticle-num.bst.

%% References without bibTeX database:

% \begin{thebibliography}{00}

%% \bibitem must have the following form:
%%   \bibitem{key}...
%%

% \bibitem{}

% \end{thebibliography}


\end{document}

%%
%% End of file 