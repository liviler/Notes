
\newpage
\section{起源}
各领域采用量子关联,对该领域问题得到解释。
\subsection{黑体辐射}
什么是黑体辐射?
Planck公式的物理意义是什么

\subsection{光电效应}
如何理解光量子

\subsection{原子结构}
Bohr将作用量子h引入Rutherford模型中,对经典物理下存在的问题得到了解释。\\
Bohr对氢原子的模型进行研究,假设:\\
1.原子能量是分立的,而非经典力学说谓的连续的。即能量只能在不同分立的态之间转换。\\
2.原子吸收或放射特定频率的能量实现在不同态之间转换。\\
基于以上原理,利用对应原理给出的氢原子能级公式,Bohr推导出氢原子从不同能级出发发生跃迁所释放的不同频率的谱线,
预测出紫外线系谱线。

\subsection{物质波}
物质粒子的波动性,使得薛定谔的波动力学能够更广泛的描述这个世界。