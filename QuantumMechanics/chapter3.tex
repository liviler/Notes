\section{算符}
\subsection{算符的性质}
量子力学中的算符就是指对波函数的一种运算,都是要作用到波函数上来体现效果。
\begin{enumerate}
    \item 线性算符:满足$\hat{A}(c_1\psi_1+c_2\psi_2)=c_1\hat{A}\psi_1+c_2\hat{A}\psi_2$的算符$\hat{A}$为线性算符\\
量子力学中刻画可观测量的算符都是线性算符。
    \item 作用结果相同的算符的算符相等。
    \item 算符之和作用效果等同于单个作用之和。且满足加法交换律和结合律。
    \item 算符之积一般不满足乘法交换律。
\end{enumerate}